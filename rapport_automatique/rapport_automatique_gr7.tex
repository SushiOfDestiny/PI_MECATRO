\documentclass{report}

\title{Mécatro: Rapport d'automatique}
\author{Groupe 7: Royer Jules, Simon Noah-Luc, Colin Matthieu, Bourderioux Armand}

\date{}

\usepackage[T1]{fontenc}
\usepackage{lmodern}

\usepackage[margin=0.7in]{geometry}

\usepackage{amssymb}
\usepackage{gensymb}
\usepackage{mathrsfs}

\usepackage{amsmath}
\usepackage{mathtools}

\usepackage{graphicx}
\usepackage{caption}
\usepackage{subcaption}
\usepackage{float}

\usepackage[parfill]{parskip}

\preto{\subsection}{\Needspace{5\baselineskip}}
\preto{\section}{\Needspace{5\baselineskip}}

\hyphenpenalty=10000

\usepackage{listings}
\usepackage{pxfonts}

\usepackage{biblatex}
\addbibresource{bibliography.bib}

\usepackage{blindtext}

\usepackage{hyperref}

\begin{document}

\maketitle

\tableofcontents

\chapter{Introduction}
Le but de la partie automatique du projet de mécatronique est de créer un contrôleur
pour un robot "bolide" suiveur de ligne. Le contrôleur a d'abord été créé théoriquement
dans le logiciel Matlab et testé sur un modèle de simulation Simulink, puis il a été
implémenté informatiquement en Arduino afin d'être embarqué dans le robot.

\nopagebreak

\chapter{Création du contrôleur théorique}

\paragraph{Contrôle d'un mouvement rectiligne uniforme}

Le mouvement du bolide auquel nous nous intéressons est rectiligne uniforme le long de l'axe $x$ du repère du laboratoire.
Nous avons espoir qu'il approxime suffisamment la dynamique locale du robot par rapport à la courbe à suivre, on
verra par la suite que cela est confirmé.
En reprenant les notations et la modélisation du document `Equations de la dynamique du Segway',
les équations de la dynamique sous forme d'état sont (on négligera toutes les perturbations):

\begin{equation*}
    \begin{cases}
        \dot{p} = u \\
        \dot{u} = \frac{1}{\beta}\big( \frac{1}{\rho}kI^{+} - m_bdv^2 \big) \\
        \dot{\psi} = v \\
        \dot{v} = \frac{1}{\gamma}\big( \frac{lk}{2\rho}I^{-} + m_bduv \big) \\
        \dot{I^{+}} = \frac{U^{+}}{L} - \frac{R}{L}I^{+} - \frac{2k}{L\rho}u \\
        \dot{I^{-}} = \frac{U^{-}}{L} - \frac{R}{L}I^{-} - \frac{kl}{L\rho}v \\
        \dot{y} = u\sin\psi \\
    \end{cases}
\end{equation*}

Les entrées sont $U^{+}$, $U^{-}$.

Les sorties mesurées sont:

\begin{equation*}
    \begin{cases}
        \delta  \phi^{+} = \delta \phi_{right} + \delta \phi_{left} = \frac{2}{\rho}p \\
        \delta  \phi^{-} = \delta \phi_{right} + \delta \phi_{left} = \frac{l}{\rho}\psi \\
        c_{LF} = \frac{y}{\cos\psi} \approx y \\
    \end{cases}
\end{equation*}

Où on a défini:

\begin{itemize}
    \item $p$ la distance curviligne parcourue par le robot le long de sa trajectoire.
    Celle-ci doit figurer dans l'état pour mesurer les angles cumulés.
    \item $u$ sa vitesse longitudinale.
    \item $\psi$ l'angle entre l'axe horizontal $x$ et la direction du robot.
    \item $I^{+} = I_{right} + I_{left}$ la somme des courants des moteurs.
    \item $I^{-} = I_{right} - I_{left}$ la différence des courants.
    \item $U^{+} = U_{right} + U_{left}$ la somme des tensions aux bornes des moteurs.
    \item $U^{-} = U_{right} - U_{left}$ la différence des tensions.
    \item $\beta = M + \frac{2I^{w}_{y}}{\rho^2}$
    \item $\gamma = I_{\psi} + m_bd^2$
    \item $\phi^{+}$ la somme des angles des roues. Posons $\alpha = \frac{2}{\rho}$.
    \item $\phi^{-}$ la différence des angles des roues. Posons  $\eta = \frac{l}{\rho}$.
    \item $c_{LF}$ la mesure de l'écart entre le point A et la ligne (approximation en MRU horizontal).
\end{itemize}


\begin{figure}[h]  % Placement "here"
    \centering
    \includegraphics[width=0.5\textwidth]{figures/cLF_schema.jpg}
    \caption{Ecart à la ligne dans l'hypothèse d'un faible angle.}
\end{figure}


\paragraph{Recherche des trajectoires d'équilibre}

Notre trajectoire d'équilibre du mouvement rectiligne uniforme 
est caractérisée par le système, avec $u_0$ 
la vitesse longiline de commande:

\begin{equation*}
    \begin{cases}
        u_0 = \overline{u} \\
        0 = \frac{1}{\beta}\big( \frac{1}{\rho}k\overline{I^{+}} - m_bd\overline{v}^2 \big) \\
        0 = \overline{v} \\
        0 = \frac{1}{\gamma}\big( \frac{lk}{2\rho}I^{-} + m_bd\overline{u}\overline{v} \big) \\
        0 = \frac{\overline{U^{+}}}{L} - \frac{R}{L}\overline{I^{+}} - \frac{2k}{L\rho}\overline{u} \\
        0 = \frac{\overline{U^{-}}}{L} - \frac{R}{L}\overline{I^{-}} - \frac{kl}{L\rho}\overline{v} \\
        0 = \overline{u}\sin\overline{\psi} \\
    \end{cases}
\end{equation*}

Avec les sorties linéarisées:

\begin{equation*}
    \begin{cases}
        \delta \phi^{+} = \alpha \delta p \\
        \delta \phi^{-} = \eta \delta \psi \\
        \delta c_{LF} = \delta y \\
    \end{cases}
\end{equation*}

Ce qui donne comme trajectoire d'équilibre 
\begin{equation*}
    \begin{cases}
        \underline{x} = \big(\overline{p}=u_0t, \overline{u}=0, \overline{\psi}=\psi_0=0, 
        \overline{v}=0, \overline{I^{\pm}}=0, \overline{y}=0 \big) \\
        \underline{e} = \big( \overline{U^{+}}=\frac{2k}{\rho} u_0, \overline{U^{-}}=0 \big)
    \end{cases}
\end{equation*}

\paragraph{Linéarisé autour de l'équilibre}

\begin{equation*}
    \begin{cases}
        \delta{\dot p} = \delta u \\
        \delta{\dot u} = \frac{1}{\beta}\frac{1}{\rho}k\delta I^{+} \\
        \delta{\dot \psi} = \delta v \\
        \delta{\dot v} = \frac{1}{\gamma}\big( \frac{lk}{2\rho}\delta I^{-} + m_bd u_0 \delta v \big) \\
        \delta{\dot I^{+}} = \frac{\delta U^{+}}{L} - \frac{R}{L}\delta I^{+} - \frac{2k}{L\rho}\delta u \\
        \delta{\dot I^{-}} = \frac{\delta U^{-}}{L} - \frac{R}{L}\delta I^{-} - \frac{kl}{L\rho}\delta v \\
        \dot{y} = u_0\cos\psi_0 \delta \psi + \delta u \sin\psi_0 = u_0 \delta \psi \\
    \end{cases}
\end{equation*}

\begin{figure}[h]  % Placement "here"
    \centering
    \includegraphics[width=0.2\textwidth]{figures/eigSys.png}
    \caption{Valeurs propres du système linéarisé.}
\end{figure}

Les valeurs propres obtenus sont négatives. Celles qui sont nulles sont
dûes au fait que certaines variables d'état sont redondantes: $p$, $y$,
donc le système est quand même stable.

% Il est étrange que les 2 valeurs propres rapides soient différentes, car
% celles de $I_l, I_r$ sont égales et que le changement en $I^+,I^-$.

\paragraph{Simplifications par perturbations singulières}

Les intensités de courant ont un transitoire bien plus rapide que
ceux des grandeurs mécaniques (leurs valeurs propres $\frac{R}{L}$ sont bien supérieures aux autres).

On applique la méthode des perturbations singulières à:
\begin{equation*}
    \begin{cases}
        L\dot{I^{+}} = U^{+} - RI^{+} - \frac{2k}{\rho}u \\
        L\dot{I^{-}} = U^{-} - RI^{-} - \frac{kl}{\rho}v \\
    \end{cases}
\end{equation*}

Avec $\varepsilon = L$ "très petit". En tout rigueur il faudrait introduire
$L_0$ pour adimensionner $\varepsilon$.

On obtient:

\begin{equation*}
    \begin{cases}
        I^{+} = \frac{1}{R} \big(U^{+} - \frac{2k}{\rho}u \big) \\
        I^{-} = \frac{1}{R} \big(U^{-} - \frac{kl}{\rho}v \big)\\
    \end{cases}
\end{equation*}

On vérifie la stabilité uniformément exponentielle de la branche d'équilibre avec :

\begin{equation*}
    \partial_{I^{\pm}}
    \begin{pmatrix}
        \frac{U^{+}}{L} - \frac{R}{L}I^{+} - \frac{2k}{L\rho}u  \\
        \frac{U^{-}}{L} - \frac{R}{L}I^{-} - \frac{kl}{L\rho}v  
    \end{pmatrix} \biggr\rvert_{I^{\pm} = f(\underline{x}, \underline{e})}
    = -\frac{R}{L}
    \begin{pmatrix}
        1 & 0 \\
        0 & 1
    \end{pmatrix}
\end{equation*}

La branche a donc comme valeurs propres des réels négatifs, donc est ue-stable.

\paragraph{Système lent simplifié}

\begin{equation*}
    \begin{cases}
        \delta{\dot p} = \delta u \\
        \delta{\dot u} = \underbrace{\frac{1}{\beta}\frac{k}{\rho}\frac{1}{R}}_{Q} \big( \delta U^+ - \frac{2k}{\rho}\delta u \big)\\
        \delta{\dot \psi} = \delta v \\
        \delta{\dot v} = \underbrace{\frac{1}{\gamma} m_bd u_0}_{b_1} \delta v + 
        \underbrace{\frac{1}{\gamma}\frac{lk}{2\rho}\frac{1}{R}}_{b_2}\big( \delta U^- - \frac{lk}{\rho}\delta v \big) \\
        \dot{y} = u_0\delta \psi        
    \end{cases}
\end{equation*}


\paragraph{Découplage en sous-système "Sum" et "Dif"}

On peut séparer le système global en 2 sous-systèmes indépendants.
Le premier, appelé "Sum", régit la dynamique du bolide
en ligne droite:

\begin{equation*}
    \text{Etat}
    \begin{cases}
        \delta{\dot p} = \delta u \\
        \delta{\dot u} = Q\big( \delta U^+ - \frac{2k}{\rho}\delta u \big)\\
    \end{cases}
    \text{Sorties}
    \begin{cases}
        \delta y_m^+ = \delta \phi^+        
    \end{cases}
\end{equation*}

Le second, appelé "Dif", régit la dynamique du bolide dans les virages:

\begin{equation*}
    \text{Etat}
    \begin{cases}
        \delta{\dot \psi} = \delta v \\
        \delta{\dot v} = b_1\delta v + b_2\big( \delta U^- - \frac{lk}{\rho}\delta v \big) \\
        \dot{y} = u_0\delta \psi        
    \end{cases}
    \text{Sorties}
    \begin{cases}
        \delta y_m^- = \delta \phi^- \\ 
        \delta c_{LF} = \delta y  
    \end{cases}
\end{equation*}

On obtient les valeurs propres suivantes:

\begin{figure}[h]  % Placement "here"
    \begin{subfigure}{.5\textwidth}
        \centering
        \includegraphics[width=0.6\textwidth]{figures/eigSysSumOl.png}        
      \end{subfigure}    
      \begin{subfigure}{.5\textwidth}
        \centering
        \includegraphics[width=0.6\textwidth]{figures/eigSysDifOl.png}
      \end{subfigure}    
      \caption{Valeurs propres des sous-systèmes linéarisés. 
      Environ égales aux plus lentes du linéarisé global.}
\end{figure}

\paragraph{Modélisation du contrôleur du sous-système "Sum"}

L'état étant de dimension 2, notre choix s'est porté sur un Proportionnel
- Dérivé permettant de contrôler $U^+$ avec $\phi^+$:

\begin{equation*}
    \delta{U^+} = h_1(\delta \phi^+_r - \delta \phi^+) 
    - h_2(\delta \dot{\phi}^+_r - \delta \dot{\phi}^+) \\     
    =h_1\alpha( \delta p_r - \delta p) 
    - h_2\alpha( \delta u_r - \delta u)
\end{equation*}
\begin{equation*}
    \text{En Laplace: }
    \underline{U^+} = \frac{-h_2s + h_1}{1}
    (\underline{\phi^+_r} -\underline{\phi^+})
\end{equation*}

Pour ajuster le comportement de notre système en boucle fermée, nous 
imposons le polynôme caractéristique de $A^+ = \partial_{\underline{x}^+} f$
où $f^+$ est la fonction de la dynamique de l'état $\underline{x}^+$:
\begin{equation*}
    \chi_{A^+ \text{ désiré}} = s^2 + 2\xi\omega_0 s + \omega_0^2
\end{equation*}

Où l'amortissement $\xi$ et la pulsation propre $\omega_0$ sont à ajuster.

En boucle fermée:

\begin{equation*}
    \begin{pmatrix}
        \delta{\dot p} \\
        \delta{\dot u} \\
    \end{pmatrix}
    =
    \underbrace{
    \begin{pmatrix}
        0 & 1 \\
        \underbrace{-Q\alpha h_1}_{a_1} & +\underbrace{Q(\alpha h_2 - \frac{2k}{\rho}}_{a_2})
    \end{pmatrix}
    }_{A^+}
    \begin{pmatrix}
        \delta{p} \\
        \delta{u} \\
    \end{pmatrix}
    +
    \begin{pmatrix}
        0 & 0 \\
        Q\alpha h_1 & -Q\alpha h_2
    \end{pmatrix}
    \begin{pmatrix}
        \delta{p_r} \\
        \delta{u_r} \\
    \end{pmatrix}
    \Rightarrow
    \chi_{A^+} = s^2 - a_2 - a_1
\end{equation*}


\begin{equation*}
    \text{Identification}
    \begin{cases}
        -a_2=2\xi \omega_0 \\
        -a_1 = \omega_0^2
    \end{cases}
    \Leftrightarrow
    \begin{cases}
        h_2 = \frac{1}{\alpha}\big( \frac{2\xi\omega_0}{-Q}+\frac{2k}{\rho})\\
        h_1 = \frac{\omega_0^2}{Q\alpha}
    \end{cases}
\end{equation*}

\paragraph{Modélisation du contrôleur du sous-système "Dif"}
Pour contrôler l'état de dimension 3, on a choisit un Proportionnel - Dérivé
sur la différence des angles cumulés $\phi^-$ 
et un proportionnel sur l'écart à la ligne $c_{LF}$ pour contrôler $U^-$:

\begin{equation*}
    \delta U^- = k_1(\delta \phi^-_r - \delta \phi^-)
    - k_2(\delta \dot{\phi}^-_r - \delta \dot{\phi}^-)
    + k_5(\delta c_{LFr} - \delta c_{LF}) \\
    = k_1 \eta(\delta \psi_r - \delta \psi^-)
    - k_2 \eta(\delta \dot{\psi}_r - \delta \dot{\psi}^-)
    + k_5(\delta y_r - \delta y) \\
\end{equation*}
\begin{equation*}
    \text{En Laplace: }
    \underline{U^-} = \frac{-k_2s + k_1}{1}
    (\underline{\phi}^-_r - \underline{\phi}^-)
    + \frac{k_5}{1}
    (\underline{c_{LFr}} - \underline{c_{LF}})    
\end{equation*}

Pour ajuster le comportement de notre système en boucle fermée, nous 
imposons:
\begin{equation*}
    \chi_{A^- \text{ désiré}} = s^3 + \sigma\sqrt{6}s^2 
    + \sigma^2\sqrt{6}s + \sigma^3 
\end{equation*}

En boucle fermée:

\begin{equation*}
    \begin{cases}
        \delta \dot{\psi} = \delta v \\
        \delta \dot{v} = \underbrace{-k_1b_2\eta\delta}_{c_1} \psi + 
        \underbrace{\big(b_1 + b_2k_2\eta - \frac{b_2kl}{\rho} \big)}_{c_2}\delta v 
        \underbrace{- b_2k_5}_{c_3} \delta y
    \end{cases}
\end{equation*}

On a

\begin{equation*}
    A^-=
    \begin{pmatrix}
        0 & 1 & 0 \\
        c_1 & c_2 & c_3 \\
        u_0 & 0 & 0 \\
    \end{pmatrix}
\end{equation*}
 On identifie
 \begin{equation*}
    \chi_{A^-} = s^3 - c_2s - c_1s - u_0c_3
    = s^3 + \sigma\sqrt{6}s^2 
    + \sigma^2\sqrt{6}s + \sigma^3 \\    
 \end{equation*}
 \begin{equation*}
    \Leftrightarrow
    \begin{cases}
        k_1 = \frac{1}{b_2\eta}\sigma^2\sqrt{6}  \\
        k_2 = \frac{1}{b_2\eta}\big(-\sigma\sqrt{6} - b_1 + b_2\frac{kl}{\rho} \big) \\
        k_5 = \frac{\sigma^3}{u_0b_3}  
    \end{cases}
 \end{equation*}

\paragraph{Approximation des contrôleurs avec des dérivées filtrées}

L'utilisation de la dérivée filtrée vient du fait que les capteurs
angulaires mesurent seulement les angles et nous avons besoin de
leur dérivée pour le contrôleur. Une simple approximation par la 
tangente n'est pas suffisante car sensible au bruit de mesure.

Pour le sous-système "Sum", le contrôleur en dérivée filtrée s'écrit:

\begin{equation*}
    \begin{cases}
        \dot{\eta} = \frac{\Delta \phi^+ - \eta}{\varepsilon}
        =  n_s\big(\Delta \phi^+ - \eta \big) 
        \text{ où } \Delta \phi^+ = \delta \phi^+_r - \delta \phi^+
        \text{ et } n_s = \frac{1}{\varepsilon} \\
        \delta{U^+} = h_1(1 - T_{ds}n_s)\Delta \phi^+ +
        h_1T_{ds}n_s\eta \text{ où } T_{ds} = \frac{h_2}{h_1} 
    \end{cases}
\end{equation*}

De même pour le sous-système "Dif":

\begin{equation*}
    \begin{cases}
        \dot{\eta} = \frac{\Delta \phi^- - \eta}{\varepsilon}
        =  n_d\big(\Delta \phi^- - \eta \big) 
        \text{ où } \Delta \phi^- = \delta \phi^-_r - \delta \phi^-
        \text{ et } n_d = \frac{1}{\varepsilon} \\
        \delta{U^-} = k_1(1 - T_{dd}n_d)\Delta \phi^- +
        k_1T_{dd}n_d\eta + k_5 \Delta c_{LF}\text{ où } T_{dd} = \frac{k_2}{k_1} \text{ et }
        \Delta c_{LF} = \delta c_{LFr} - \delta c_{LF}
    \end{cases}
\end{equation*}

\paragraph{Réponses fréquentielles des systèmes fermés modélisés 
en Matlab}

Pour le système "Sum", nous ne voulions pas trop accélérer la dynamique,
il ne nous semblait pas essentielle d'avoir une réponse extrêmement
rapide pour atteindre la vitesse de consigne, par rapport au
suivi de la ligne, géré par l'autre contrôleur.
Ainsi nous avons pris $\omega_0 = |\lambda^+|$ 
valeur absolue de la valeur propre du système Sum,
et l'amortissement $\xi = \frac{1}{\sqrt{2}}$, afin d'avoir en boucle fermée
(avec $n_s=150$):

\begin{figure}[h]  % Placement "here"
    \begin{subfigure}{.5\textwidth}
        \centering
        \includegraphics[width=0.8\textwidth]{figures/eigSysSumBF.png}        
      \end{subfigure}    
      \begin{subfigure}{.5\textwidth}
        \centering
        \includegraphics[width=0.8\textwidth]{figures/eigSysSumBFApprox.png}
      \end{subfigure}    
      \caption{Valeurs propres de "Sum" avec contrôleur PD, 
      (gauche: contrôleur idéal, droite: dérivée filtrée).
      Leurs valeurs absolues sont sensiblement égales à
      celle en boucle ouverte.
      Celle du filtre est plus grande.}
\end{figure}


\begin{figure}[h]  % Placement "here"
    \begin{subfigure}{.4\textwidth}
        \centering
        \includegraphics[width=\textwidth]{figures/reptempsumbf_approx.jpg}        
      \end{subfigure}    
      \begin{subfigure}{.6\textwidth}
        \centering
        \includegraphics[width=\textwidth]{figures/repfreqsumbf_approx.jpg}
      \end{subfigure}    
      \caption{Réponses temporelles et fréquentielles 
      du système "Sum" contrôlé (à un échelon de la somme des angles).
      Avec le contrôleur avec dérivée filtrée, la 
      réponse oscille un peu et est en retard.
      La réponse sur $u$ à une marge de phase de 180° et une 
      marge de gain infinie ce qui est satisfaisant.}
\end{figure}

Pour le sous-système "Dif", nous avons accéléré la dynamique avec
$\sigma = 15$ ($n_d=150$).

\begin{figure}[h]  % Placement "here"
    \begin{subfigure}{.5\textwidth}
        \centering
        \includegraphics[width=0.8\textwidth]{figures/eigSysDifBF.png}        
      \end{subfigure}    
      \begin{subfigure}{.5\textwidth}
        \centering
        \includegraphics[width=0.8\textwidth]{figures/eigSysDifBFApprox2.png}
      \end{subfigure}    
      \caption{Valeurs propres de "Dif" avec contrôleur PD+P, 
      (gauche: contrôleur idéal, droite: dérivée filtrée).
      Leurs valeurs absolues sont sensiblement égales à celles en boucle ouverte, 
      ce qui montre que la dynamique n'a pas vraiment été accélérée, et celles du sous-système
      approché par la dérivée filtrée approximent assez bien celle du système
      théorique en boucle fermée. 
      La valeur propre du filtre est 7 fois plus rapide que les autres.}
\end{figure}

\begin{figure}[h]  % Placement "here"
    \begin{subfigure}{.5\textwidth}
        \centering
        \includegraphics[width=\textwidth]{figures/reptempdifbf_approx.jpg}        
      \end{subfigure}    
      \begin{subfigure}{.5\textwidth}
        \centering
        \includegraphics[width=\textwidth]{figures/repfreqdifbf_approx.jpg}
      \end{subfigure}    
      \caption{
        Réponses temporelles et fréquentielles 
      du système "Dif" contrôlé.
      La réponse de $y$ après un échelon sur $cLF_r$ s'effectue 
      sans oscillation ni dépassement en environ $0.6s$ ce qui est 
      satisfaisant par rapport aux attentes de réactivité d'un bolide.
      L'échelon sur la différence des angles n'est pas vraiment notable.
      La marge de gain de la réponse de $y$ à un échelon de $\phi^-_r$
      est de $60 dB$, et celle en phase est infinie.
      Tandis que la marge de gain de la réponse de $y$ à 
      un échelon de $cLF_r$ est de $16 dB$ (satisfaisant) et celle en phase est nulle, 
      ce qui n'est pas satisfaisant.}
\end{figure}

\newpage

\paragraph{Etude sur le système de simulation Matlab}

Le fichier de Matlab s'intitule `plant2MRUEBF.slx'.

\begin{figure}[h]
    \centering
    \includegraphics[width=\textwidth]{figures/sysSimulBF.png}
    \caption{
    Schéma du modèle Matlab de simulation contrôlé. 
    Les pertubations de tension sont mises à zéro.
    Un bloc de discrétisation est ajouté en fin de mesure du capteur
    de ligne $cLF$.}
\end{figure}

Nous étudions la réponse en poursuite, ie le régime transitoire d'un
état initial perturbé par un écart à la ligne de $\delta y = 3 \text{ cm}$.
La vitesse consigne est $0.3 \text{ m/s}$.


\begin{figure}[h]  % Placement "here"
    \centering
    \begin{subfigure}{\textwidth}
        \centering
        \includegraphics[width=0.7\textwidth]{figures/sysSimulBFrepInputs.png}        
      \end{subfigure}   
    \newline
      \begin{subfigure}{\textwidth}
        \centering
        \includegraphics[width=\textwidth]{figures/sysSimulBFrepOutputs.png}
      \end{subfigure}    
      \caption{
    Tension aux bornes des moteurs (à gauche) 
    et sorties mesurées (à droite), on note une oscillation notable
    de la différence des tension, est une valeur presque constante 
    de leur somme de $\approx 6V$. 
    On observe d'autre part une rapide décroissance de la 
    mesure du capteur, sans oscillation, jusqu'à une position proche de la position
    d'équilibre, ce qui est satisfaisant.
    En considérant cet écart statique, nous nous sommes demandés si cela présentait un
    intérêt d'ajouter un terme intégral en $c_{LF}$ pour l'annuler.
    Sachant que les mesures du capteur sont discrètes, nous avons préféré nous en passer,
    car l'erreur de l'ordre du $mm$ est mesurée comme nulle sachant que la résolution
    du capteur est $0.09 / 7 m >> 1 mm $. }
\end{figure}



\chapter{Identification des paramètres physiques}

Nous avons rassemblé toutes les valeurs numériques dans ce 
\href{https://docs.google.com/spreadsheets/d/1PVCPAeFXgacQK3YaMxcYYtDlWB_0VT0cpRnJDznpAQc/edit?pli=1#gid=0
}{Google Sheet}.

\paragraph{Estimation de la constante de couple k}

Les équations électriques à l'équilibre s'écrivent

\begin{equation*}
    \begin{cases}
        0 = U - RI - k\Omega \\
        \tau = kI       
    \end{cases}
\end{equation*}

En se plaçant sans couple, donc sans courant, 
et en utilisant les courbes constructeurs, on calcule
$k=\frac{U}{\Omega} \, V/(rad.s^-1)$.

\begin{figure}[h]  % Placement "here"
    \centering
    \includegraphics[width=9cm]{figures/courbes-construc.png}


    \caption{Courbes de fonctionnement fournies par Pololu.}
\end{figure}

\paragraph{Estimation des masses des pièces}
Toutes les pièces ont été pesées.

On a approximé la masse de l'arbre moteur et du rotor à la différence
entre la masse pesée du moteur, et celle donnée par la CAO, car on a supposé que cette dernière 
correspondait uniquement à la structure extérieure du moteur, ie au stator.

\paragraph{Estimation des moments d'inertie}

Pour le moment d'inertie de l'unité roue + engrenage + rotor, 
notre première approche (et celle que nous avons gardée pour 
les simulations) a été d'approximer la géométrie du système à celle d'un disque,
et d'utiliser la matrice d'inertie (cellule D34 du GSheet):

\begin{equation*}
    I_w(C_w) = 
    \begin{pmatrix}
        m \big(\rho^2/4 + l_w^2/2 \big) & 0 & 0 \\
        0 & m \big(\rho^2/4 + l_w^2/2 \big) & 0 \\
        0 & 0 & m \big(R^2/4 + l^2/12 \big)
    \end{pmatrix}
\end{equation*}

Comme autre approche, nous avons mesuré le temps de réponse de la 
vitesse angulaire de la roue à un échelon de 12V aux bornes du moteur.
Nous avons mesuré la sommes des positions angulaires, que nous avons filtrée 
avec un filtre passe bas pour en obtenir une estimation de la 
dérivée, en supposant que cela ne fausse pas sensiblement le 
temps de réponse.

\begin{figure}[h]  % Placement "here"
    \centering
    \includegraphics[width=15cm]{figures/inertie_roue.png}
    \caption{Vitesse angulaire (degrés/s) de la roue en fonction du temps (s).}
\end{figure}

Les données constructeurs fournissent en effet une relation 
linéaire entre la vitesse angulaire et le couple d'entrée. $$\omega = a - b\tau$$

On a essayé d'exploiter le théorème du moment cinétique à l'unité roue+rotor:

$$I_w^y\dot{\omega} = \tau = \frac{a}{b} - \frac{\omega}{b}$$
$$\dot{\omega} +  \frac{\omega}{T} = \frac{a}{T} \text{  où  } T=I_w^yb$$
$$\text{De solution: } \omega = a\left(1 - \exp\left(-\frac{t}{T}\right) \right)$$

On trouve $T$ comme l'abscisse de la courbe quand $y=a\left(1 - e^{-1}\right) \approx 0.63a$. 
Cependant l'asymptote mesurée vaut $a_{exp} = 200 \text{ degrés}.s^{-1} = 3.5 rad.s^{-1}$ 
ce qui est 10 fois plus lent que la valeur constructeur: $a_{th} = 320 \text{ rpm} = 33.5 rad.s^{-1}$

On trouve alors en abscisse de $\omega = 0.63\cdot200 \text{ degrés}.s^{-1}$ : $T=0.1s$ (cellule B56) 
puis $I_w^y = 44.2 kg.m^2$ différent d'un ordre de grandeur de $10^5$ des valeurs données par la CAO,
ce qui est aberrant.

Enfin notre dernière approche a été la CAO, a l'aide de des estimations des masses (cellule F35).
Les valeurs obtenues sont cette fois plus petite d'un facteur $100$.

Il est suprenant et frustrant d'avoir tant de variance dans les mesures des paramètres.


\paragraph{Estimation des constantes électriques}
La résistance et l'inductance interne du moteur ont été trouvées sur Internet 
\href{https://forum.pololu.com/t/mechanics-and-electrical-parameters/18153/2}{ici}.
Par manque de temps, nous n'avons pas comparé cela à des mesures sur notre bolide.

\chapter{Implémentation du contrôleur en Arduino}

% \printbibliography

\end{document}
